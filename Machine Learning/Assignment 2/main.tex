\documentclass[a4paper]{article}
\usepackage{bm}
\usepackage[leqno]{amsmath}
\usepackage{mathtools}
\usepackage{amssymb}

\makeatletter
\newcommand{\leqnomode}{\tagsleft@true}
\newcommand{\reqnomode}{\tagsleft@false}
\makeatother

%!TeX root=main.tex

\addtolength{\hoffset}{-2.25cm}
\addtolength{\textwidth}{4.5cm}
\addtolength{\voffset}{-3.25cm}
\addtolength{\textheight}{5cm}
\setlength{\parskip}{0pt}
\setlength{\parindent}{0in}

%----------------------------------------------------------------------------------------
%	PACKAGES AND OTHER DOCUMENT CONFIGURATIONS
%----------------------------------------------------------------------------------------

\usepackage{blindtext} % Package to generate dummy text
\usepackage{charter} % Use the Charter font
\usepackage[utf8]{inputenc} % Use UTF-8 encoding
\usepackage{microtype} % Slightly tweak font spacing for aesthetics
\usepackage[english, ngerman]{babel} % Language hyphenation and typographical rules
\usepackage{amsthm, amsmath, amssymb} % Mathematical typesetting
\usepackage{float} % Improved interface for floating objects
\usepackage[final, colorlinks = true, 
            linkcolor = black, 
            citecolor = black]{hyperref} % For hyperlinks in the PDF
\usepackage{graphicx, multicol} % Enhanced support for graphics
\usepackage{xcolor} % Driver-independent color extensions
\usepackage{marvosym, wasysym} % More symbols
\usepackage{rotating} % Rotation tools
\usepackage{censor} % Facilities for controlling restricted text
\usepackage{listings, style/lstlisting} % Environment for non-formatted code, !uses style file!
\usepackage{pseudocode} % Environment for specifying algorithms in a natural way
\usepackage{style/avm} % Environment for f-structures, !uses style file!
\usepackage{booktabs} % Enhances quality of tables
\usepackage{tikz-qtree} % Easy tree drawing tool
\tikzset{every tree node/.style={align=center,anchor=north},
         level distance=2cm} % Configuration for q-trees
\usepackage{style/btree} % Configuration for b-trees and b+-trees, !uses style file!
\usepackage[backend=biber,style=numeric,
            sorting=nyt]{biblatex} % Complete reimplementation of bibliographic facilities
\addbibresource{ecl.bib}
\usepackage{csquotes} % Context sensitive quotation facilities
\usepackage[yyyymmdd]{datetime} % Uses YEAR-MONTH-DAY format for dates
\renewcommand{\dateseparator}{-} % Sets dateseparator to '-'
\usepackage{fancyhdr} % Headers and footers
\pagestyle{fancy} % All pages have headers and footers
\fancyhead{}\renewcommand{\headrulewidth}{0pt} % Blank out the default header
\fancyfoot[L]{} % Custom footer text
\fancyfoot[C]{} % Custom footer text
\fancyfoot[R]{\thepage} % Custom footer text
\newcommand{\note}[1]{\marginpar{\scriptsize \textcolor{red}{#1}}} % Enables comments in red on margin

%----------------------------------------------------------------------------------------

\begin{document}

%-------------------------------
%	TITLE SECTION
%-------------------------------

\fancyhead[C]{}
\hrule \medskip % Upper rule
\begin{minipage}{0.295\textwidth} 
\raggedright
\footnotesize
Julien Fer \hfill\\   
10649441\hfill\\
julien.fer88@gmail.com
\end{minipage}
\begin{minipage}{0.4\textwidth} 
\centering 
\large 
Homework Assignment 1\\ 
\normalsize 
Machine Learning 1, 19/20\\ 
\end{minipage}
\begin{minipage}{0.295\textwidth} 
\raggedleft
\today\hfill\\
\end{minipage}
\medskip\hrule 
\bigskip

%-------------------------------
%	CONTENTS
%-------------------------------

\section{Basic Linear Algebra and Derivatives}
This part was not required to hand in.
% \subsection{First Subtask}
% Some equations
% \begin{align*}
% y &=  \sum\limits_{i,k} m_i \cdot f^k \\
% x &=  
% \underset{11}{\underbrace{3 + 8}} + 5 + 7
% \end{align*}

% \subsection{Second Subtask}
% \blindtext

% \bigskip

%------------------------------------------------

\section{Multivariate Calculus}
\subsection{Question 2.1}
\textbf{a)} The following gradient is asked: $\nabla_{\bm{\mu}} (\textbf{x} - \bm{\mu})^{T} \bm{\Sigma}^{-1} (\textbf{x} - \bm{\mu})$. In order to find the gradient the following variables are defined:
\begin{flalign}
\textbf{y}(\bm{\mu}) &\coloneqq \textbf{x}-\bm{\mu} &&
\end{flalign}
{\setlength{\abovedisplayskip}{0pt}%
\begin{flalign}
\textbf{f}(\textbf{y}(\bm{\mu})) &\coloneqq \textbf{y}^{T} \bm{\Sigma}^{-1} \textbf{y} &&
\end{flalign}%
}%
and thus by combining (1) and (2) in the original equation we get:
\begin{flalign}
\nabla_{\bm{\mu}} &(\textbf{x} - \bm{\mu})^{T} \bm{\Sigma} (\textbf{x} - \bm{\mu}) = \nabla_{\bm{\mu}} \textbf{f} = \frac{\partial \textbf{f}}{\partial \textbf{y}} \frac{\partial \textbf{y}}{\partial \bm{\mu}}~~\text{with} &&
\end{flalign}
{\setlength{\abovedisplayskip}{0pt}%
\begin{flalign}
\frac{\partial \textbf{f}}{\partial \textbf{y}} &= \textbf{y}^{T} (\bm{\Sigma}^{-1} + (\bm{\Sigma}^{-1})^{T}) = 2\textbf{y}^{T} \bm{\Sigma}^{-1}~\text{(as $\bm{\Sigma}^{-1}$ is symmetric)} && \\
\frac{\partial \textbf{y}}{\partial \bm{\mu}} &= -\textbf{\textit{I}}
\end{flalign}%
}%
and by finally combining (4) and (5) in (3) we get the gradient:
\begin{flalign*}
\nabla_{\bm{\mu}} \textbf{f} = \frac{\partial \textbf{f}}{\partial \textbf{y}} \frac{\partial \textbf{y}}{\partial \bm{\mu}} =  -2\textbf{y}^{T} \bm{\Sigma}^{-1} \textbf{\textit{I}} = -2(\textbf{x} - \bm{\mu})^{T}\bm{\Sigma}^{-1} &&
\end{flalign*}

\bigskip

\textbf{b)} The derivation of the gradient is as follows:
\begin{flalign*}
\nabla_{\textbf{q}} -\textbf{p}^{T} \text{log}(\textbf{q}) &= -\textbf{p}^{T} \nabla_{\textbf{q}} \text{log}(\textbf{q}) \\ 
&= -\textbf{p}^{T}  
\begin{bmatrix}
\frac{\partial \text{log}(q_{1})}{\partial q_{1}}  & \dots & \frac{\partial \text{log}(q_{1})}{\partial q_{n}}  \\
\vdots  & \ddots  & \vdots \\
\frac{\partial \text{log}(q_{n})}{q_{1}} & \ddots   & \frac{\partial \text{log}(q_{n})}{\partial q_{n}}
\end{bmatrix} \\
&= -\textbf{p}^{T}
\begin{bmatrix}
\frac{1}{\partial q_{1}}  & \dots & 0  \\
\vdots  & \ddots  & \vdots \\
0 & \ddots   & \frac{1}{\partial q_{n}}
\end{bmatrix} \\
&= -\begin{bmatrix}
\frac{p_{1}}{q_{1}} & \dots & \frac{p_{n}}{q_{n}}
\end{bmatrix}
&&
\end{flalign*}

\bigskip

\textbf{c)} We want $ \nabla_{\textbf{W}} \textbf{f} $, where $ \textbf{f} = \textbf{Wx}$ with $ \textbf{W} \in \mathbb{R}^{2 \times 3} $ and $ \textbf{x} \in \mathbb{R}^{3} $. According to example 5.11 of the textbook MML the dimension of the gradient  $\nabla_{\textbf{W}} \textbf{f}$ will be $\mathbb{R}^{2 \times (2 \times 3)}$. Thus we have
\begin{flalign}
\nabla_{\textbf{W}} \textbf{f} &= \begin{bmatrix}
\frac{\partial f_{1}}{\partial \textbf{W}} \\
\frac{\partial f_{2}}{\partial \textbf{W}}
\end{bmatrix} = \begin{pmatrix}
\begin{bmatrix}
\frac{\partial f_{1}}{\partial \textbf{W}_{1}} \\
\frac{\partial f_{1}}{\partial \textbf{W}_{2}}
\end{bmatrix} \\
\begin{bmatrix}
\frac{\partial f_{2}}{\partial \textbf{W}_{1}} \\
\frac{\partial f_{2}}{\partial \textbf{W}_{1}}
\end{bmatrix}
\end{pmatrix}~ \text{with}~ f_{i} = \sum_{j=1}^{3} W_{ij} x_{j} ~~~i = 1, 2
&& 
\end{flalign}
and for the partial derivatives we get:
\begin{flalign}
\frac{\partial f_{i}}{\partial W_{iq}} &= x_{q} &&
\end{flalign}
This results in:
\begin{flalign}
\frac{\partial f_{i}}{\partial \textbf{W}_{i}} &= \textbf{x}^{T} \in \mathbb{R}^{1 \times 1 \times 3}~~ \text{and} ~~\frac{\partial f_{i}}{\partial \textbf{W}_{j \neq i}} = \textbf{0}^{T}  \in \mathbb{R}^{1 \times 1 \times 3} &&
\end{flalign}
From the result in (8) together with (6) we can construct the gradient:
\begin{flalign*}
\nabla_{\textbf{W}} \textbf{f} &= \begin{bmatrix}
\frac{\partial f_{1}}{\partial \textbf{W}} \\
\frac{\partial f_{2}}{\partial \textbf{W}}
\end{bmatrix} = \begin{pmatrix}
\begin{bmatrix}
\textbf{x}^{T} \\
\text{0}^{T}
\end{bmatrix} \\
\begin{bmatrix}
\textbf{0}^{T} \\
\textbf{x}^{T}
\end{bmatrix}
\end{pmatrix} &&
\end{flalign*}

\bigskip

\textbf{d)} The derivation of the gradient is as follows:
\begin{flalign*}
\nabla_{\textbf{W}} \textbf{f} &= \nabla_{\textbf{W}} \bigl((\bm{\mu} - \textbf{Wx})^{T} \bm{\Sigma}^{-1} (\bm{\mu} - \textbf{Wx})\bigr) \\
&= \nabla_{\textbf{W}} \bigl((\bm{\mu}^{T} - \textbf{x}^{T}\textbf{W}^{T}) \bm{\Sigma}^{-1} (\bm{\mu} - \textbf{Wx})\bigr) \\
&= \nabla_{\textbf{W}} (\bm{\mu}^{T}\bm{\Sigma}^{-1}\bm{\mu} - \bm{\mu}^{T}\bm{\Sigma}^{-1}\textbf{Wx} - \textbf{x}^{T}\textbf{W}^{T}\bm{\Sigma}^{-1}\bm{\mu} + \textbf{x}^{T}\textbf{W}^{T}\bm{\Sigma}^{-1}\textbf{Wx}) \\
&= 0 -\bm{\Sigma}^{-1}\bm{\mu}\textbf{x}^{T} -\bm{\Sigma}^{-1}\bm{\mu}\textbf{x}^{T} + \bm{\Sigma}^{-1}\textbf{Wx}\textbf{x}^{T} \\
&= -2 \bm{\Sigma}^{-1}(\bm{\mu} - \textbf{Wx})\textbf{x}^{T}
&&
\end{flalign*}

\bigskip

%------------------------------------------------

\section{Probability Theory}
\subsection{Question 3.1}

\textbf{a)} Based on an individual's experience with criminal activities and the given circumstances (observation) it seems more likely (higher probability) that the man is a criminal.

\bigskip

\textbf{b)} Define the following variables:
\begin{flalign*}
C &= 
\begin{cases}
1, & \text{if}\ \textit{criminal} \\
0, & \textit{otherwise}
\end{cases} 
&&
\end{flalign*}
{\setlength{\abovedisplayskip}{0pt}%
\begin{flalign*}
O &= 
\begin{cases}
1, & \text{if}\ \textit{observation is made} \\
0, & \textit{otherwise}
\end{cases} 
&&
\end{flalign*}%
}%
and thus the probability, based on our beliefs of making the observation, that the man is a criminal can be formulated as follows:
\begin{flalign}
p(C = 1 \mid O = 0) &= \frac{p(O = 0 \mid C = 1) p(C = 1)}{p(O = 0)} \\
p(C = 1 \mid O = 1) &= \frac{p(O = 1 \mid C = 1) p(C = 1)}{p(O = 1)}
&&
\end{flalign}

\bigskip

\textbf{c)} We have the following probabilities:
\begin{flalign*}
p(C = 1) &= \frac{1}{10^{5}} \\
p(C = 0) &= 1 - \frac{1}{10^{5}} \\
p(O = 1 \mid C = 0) &= \frac{1}{10^{6}} \\
p(O = 1 \mid C = 1) &= 0.8
&&
\end{flalign*}
and we get the probability of the man being a criminal given the described observation using Bayes' rule (equation 10):
\begin{flalign*}
p(C = 1 \mid O = 1) &= \frac{p(O = 1 \mid C = 1) p(C = 1)}{p(O = 1)} \\
&= \frac{p(O =1 \mid C = 1) p(C = 1)}{p(O =1 \mid C = 0) p(C = 0) + p(O = 1 \mid C = 1) p(C = 1)}~~\text{(product rule)} \\
&= \frac{0.8 \cdot \frac{1}{10^{5}}}{\frac{1}{10^{6}} \cdot (1 - \frac{1}{10^{5}}) + 0.8 \cdot \frac{1}{10^{5}}} = \frac{8}{9} \approx 0.89
&&
\end{flalign*}

\bigskip

\textbf{d)} Given the new information it seems more reasonable to assume that, for example, the shopowner entered to protect its belongings. Hence, the observation will probably made more often than before, and thus increasing $p(O = 1)$. This in turn decreases the belief the man is a criminal when the observation is made ($p(C = 1 \mid O = 1)$) 

\bigskip

\subsection{Question 3.2}

\textbf{a)} From Bishop 2.29 we get the following likelihood:
{\setlength{\abovedisplayskip}{0pt}%
\begin{flalign*}
p(D \mid \rho) &= \prod_{n=1}^{N} \prod_{k=1}^{K} \rho_{k}^{x_{nk}}  = \prod_{k=1}^{K} \rho_{k}^{\sum_{n=1}^{N} x_{nk}} = \prod_{k=1}^{4} \rho_{k}^{\sum_{n=1}^{N} x_{nk}} = \prod_{k=1}^{4} \rho_{k}^{m_{k}}
&&
\end{flalign*}%
}%
with $m_{k} = \sum_{n} x_{nk}$

\bigskip

\textbf{b)} From Bishop 2.33 and with $m_{k}$ the same as in exercise a) we get the following maximum likelihood estimator:
%{\setlength{\abovedisplayskip}{0pt}%
\begin{flalign*}
\rho_{k}^{ML} &= \frac{m_{k}}{N}
&&
\end{flalign*}
%}%
and thus we have the following maximum likelihood solution:
{\setlength{\abovedisplayskip}{0pt}%
\begin{flalign*}
\rho_{ML} &= \begin{bmatrix}
\frac{4}{8}, & 0, & 0, & \frac{4}{8}
\end{bmatrix} = \begin{bmatrix}
\frac{1}{2}, & 0, & 0, & \frac{1}{2}
\end{bmatrix}
&&
\end{flalign*}%
}%

\bigskip

\textbf{c)} From Bishop 2.5 we get the following likelihood:
{\setlength{\abovedisplayskip}{0pt}%
\begin{flalign*}
p(D \mid \rho) &= \prod_{n=1}^{N} \rho^{x_{n}} (1 - \rho)^{1 - x_{n}}
&&
\end{flalign*}%
}%

\bigskip

\textbf{d)} From Bishop 2.7 we get the following maximunm likelihood estimation:
{\setlength{\abovedisplayskip}{0pt}%
\begin{flalign*}
\rho_{ML} &= \frac{1}{N} \sum_{n=1}^{N} x_{n} = \frac{6}{8} = 0.75
&&
\end{flalign*}%
}%

\bigskip

\textbf{e)} The probability of drawing a red card is equal to the sumation of the probability to draw a card with suit heart or diamonds: $\rho = \rho_{1} + \rho_{3}$

\bigskip

\textbf{f)} The Bayes rule is given by:
{\setlength{\abovedisplayskip}{0pt}%
\begin{flalign*}
\overbrace{p(\rho \mid D)}^{posterior} &= \frac{\overbrace{p(D \mid \rho)}^{likelihood} \overbrace{p(\rho)}^{prior}}{\underbrace{p(D)}_{evidence}}
&&
\end{flalign*}%
}%

\bigskip

\textbf{g)} The derivation of the MAP estimator is as follows:
{\setlength{\abovedisplayskip}{0pt}%
\begin{flalign*}
p(\rho \mid D) &= \frac{p(D \mid \rho) p(\rho)}{p(D)} \\
&\propto p(D \mid \rho)p(\rho) ~~~\text{(as $p(D)$ does not depend on $\rho$)} \\
&= \prod_{n=1}^{N} \rho^{x_{n}} (1 - \rho)^{1 - x_{n}} \frac{\Gamma(\alpha + \beta)}{\Gamma(\alpha) \Gamma(\beta)} \rho^{\alpha - 1} (1 - \rho)^{\beta - 1} \\
&\propto \prod_{n=1}^{N} \rho^{x_{n}} (1 - \rho)^{1 - x_{n}} \rho^{\alpha - 1} (1 - \rho)^{\beta - 1}~~~\text{(as the $\Gamma()$ functions do not depend on $\rho$)} \\
&= \rho^{\sum_{n=1}^{N} x_{n}} (1 - \rho)^{N - \sum_{n=1}^{N} x_{n}} \rho^{\alpha - 1} \rho^{\beta - 1} \\
&= \rho^{\sum_{n=1}^{N} x_{n} + \alpha - 1} (1 - \rho)^{N - \sum_{n=1}^{N} x_{n} +\beta - 1} \\
&= \rho^{m + \alpha - 1} (1 - \rho)^{N - m + \beta -1}
&&
\end{flalign*}%
}%
with $m = \sum_{n=1}^{N} x_{n}$. Subsequently we determine the derivative of the logarithm of the likelihood and set this equal to 0 (to find the optimal solution):
\begin{flalign*}
\frac{\partial \log(p(\rho \mid D))}{\partial \rho} &=	\frac{\partial}{\partial \rho} \bigl((m + \alpha - 1) \log(\rho) + (N - m + \beta -1) \log(1 - \rho)\bigr) \\
&= \frac{m + \alpha - 1}{\rho} - \frac{N - m + \beta - 1}{1 - \rho} = 0 \\
&\Leftrightarrow \frac{m + \alpha - 1}{\rho} = \frac{N - m + \beta - 1}{1 - \rho} \implies m + \alpha - 1 = N\rho + \alpha\rho + \beta\rho - 2\rho \\
&\implies \rho_{MAP} = \frac{m + \alpha - 1}{N + \alpha + \beta - 2}
&&
\end{flalign*}
As $m$ and $N$ are equal to each other this does not influence the parameter $\rho$. Thus to ensure an equal probability for drawing a red or black card one needs to solve the following equation:
\begin{flalign*}
\rho &= \frac{\alpha - 1}{\alpha + \beta - 2} = \frac{1}{2} \implies 2\alpha - 2 = \alpha + \beta - 2 \implies \alpha = \beta
&&
\end{flalign*}


\end{document}
